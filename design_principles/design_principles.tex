\documentclass[xcolor=svgnames]{beamer}

\usepackage[utf8]    {inputenc}
\usepackage[T1]      {fontenc}
\usepackage[english] {babel}
\usepackage{tikz}
\usepackage{amsmath,amsfonts,graphicx}
\usepackage{beamerleanprogress}


\title
    [Design Principles\hspace{2em}]
    {Design Principles and Design Patterns}

\author
    [Ryan Bartling]
    {D. Ryan Bartling}

\date
{2018-05-14}

\AtBeginSection[]
{
    \begin{frame}<beamer>
        \frametitle{Outline for section \thesection}
        \tableofcontents[currentsection]
    \end{frame}
}

\begin{document}

\maketitle

\section{Symptoms of Rotting Design}

%%%%%%%%%%%%%%%%%%%%%%%%%%%%%%%%%%%%%%%%%%%%%%%%%%%%%%%%%%%%%%%%%%%%%%%%%%%%%%%%

\begin{frame}
    {Symptoms of Rotting Code}

    Four Symptoms of Rotting Code\pause

    \begin{enumerate}
        \item Rigidity\pause
        \item Fragility\pause
        \item Immobility\pause
        \item Viscosity\pause
    \end{enumerate}
\end{frame}

%%%%%%%%%%%%%%%%%%%%%%%%%%%%%%%%%%%%%%%%%%%%%%%%%%%%%%%%%%%%%%%%%%%%%%%%%%%%%%%%

{%
\usebackgroundtemplate{%
    \tikz\node[opacity=1.0]{%
        \includegraphics<handout:0>[height=\paperheight,width=\paperwidth]{%
            brick.jpg}};}%
\begin{frame}<handout:0>{Rigidity}
\end{frame}
}

{%
\usebackgroundtemplate{%
    \tikz\node[opacity=0.2]{%
        \includegraphics<handout:0>[height=\paperheight,width=\paperwidth]{%
            brick.jpg}};}%
\begin{frame}{Rigidity}

    \begin{itemize}
        \item \href{https://www.merriam-webster.com/dictionary/rigid}
            {Deficient in or devoid of flexibility}
        \item<2-> Software for which extra effort is expended in order to make
            changes.
    \end{itemize}
\end{frame}
}

%%%%%%%%%%%%%%%%%%%%%%%%%%%%%%%%%%%%%%%%%%%%%%%%%%%%%%%%%%%%%%%%%%%%%%%%%%%%%%%%

{%
\usebackgroundtemplate{%
    \tikz\node[opacity=0.2]{%
        \includegraphics<handout:0>[height=\paperheight,width=\paperwidth]{%
            brick.jpg}};}%
\begin{frame}{Rigidity}

    How it happens
    \begin{itemize}
        \item<1-> Code written in a procedural way
        \item<2-> Lack of abstractions
        \item<3-> Solving a generic problem with implementation specific details
        \item<4-> Spreading a single responsibility throughout several parts
        \item<5-> When components need a lot of knowledge about each other in
            order to function
    \end{itemize}
\end{frame}
}

%%%%%%%%%%%%%%%%%%%%%%%%%%%%%%%%%%%%%%%%%%%%%%%%%%%%%%%%%%%%%%%%%%%%%%%%%%%%%%%%

{%
\usebackgroundtemplate{%
    \tikz\node[opacity=0.2]{%
        \includegraphics<handout:0>[height=\paperheight,width=\paperwidth]{%
            brick.jpg}};}%
\begin{frame}<handout:0>{Rigidity}
    \centering
    \includegraphics[height=0.1\textheight]{rigid1}
\end{frame}
}

{%
\usebackgroundtemplate{%
    \tikz\node[opacity=0.2]{%
        \includegraphics<handout:0>[height=\paperheight,width=\paperwidth]{%
            brick.jpg}};}%
\begin{frame}<handout:0>{Rigidity}
    \centering
    \includegraphics[height=0.1\textheight]{rigid2}
\end{frame}
}

{%
\usebackgroundtemplate{%
    \tikz\node[opacity=0.2]{%
        \includegraphics<handout:0>[height=\paperheight,width=\paperwidth]{%
            brick.jpg}};}%
\begin{frame}<handout:0>{Rigidity}
    \centering
    \includegraphics[height=0.5\textheight]{rigid3}
\end{frame}
}

{%
\usebackgroundtemplate{%
    \tikz\node[opacity=0.2]{%
        \includegraphics<handout:0>[height=\paperheight,width=\paperwidth]{%
            brick.jpg}};}%
\begin{frame}{Rigidity}
    \centering
    \includegraphics[height=0.9\textheight]{rigid}
\end{frame}
}

%%%%%%%%%%%%%%%%%%%%%%%%%%%%%%%%%%%%%%%%%%%%%%%%%%%%%%%%%%%%%%%%%%%%%%%%%%%%%%%%

{%
\usebackgroundtemplate{%
    \tikz\node[opacity=0.2]{%
        \includegraphics<handout:0>[height=\paperheight,width=\paperwidth]{%
            brick.jpg}};}%
\begin{frame}{Rigidity}

    How to avoid it
    \begin{itemize}
        \item<1-> Break the code into smaller concepts
        \item<2-> Solve the details and provide a problem oriented abstraction
        \item<3-> Solving a generic problem with implementation specific details
        \item<4-> Write DRY code (Don't repeat yourself)
        \item<5-> Define the code in logical pieces.  Set boundaries and
            responsibilities.
    \end{itemize}
\end{frame}
}

%%%%%%%%%%%%%%%%%%%%%%%%%%%%%%%%%%%%%%%%%%%%%%%%%%%%%%%%%%%%%%%%%%%%%%%%%%%%%%%%

{%
\usebackgroundtemplate{%
    \tikz\node[opacity=1.0]{%
        \includegraphics<handout:0>[height=\paperheight,width=\paperwidth]{%
            glass.jpg}};}%
\begin{frame}<handout:0>{Fragility}
\end{frame}
}

{%
\usebackgroundtemplate{%
    \tikz\node[opacity=0.2]{%
        \includegraphics<handout:0>[height=\paperheight,width=\paperwidth]{%
            glass.jpg}};}%
\begin{frame}{Fragility}

    \begin{itemize}
        \item \href{https://www.merriam-webster.com/dictionary/fragile}
            {Easily broken or destroyed}
        \item<2-> Software for which extra risk is incurred in order to make
            changes.
    \end{itemize}
\end{frame}
}

%%%%%%%%%%%%%%%%%%%%%%%%%%%%%%%%%%%%%%%%%%%%%%%%%%%%%%%%%%%%%%%%%%%%%%%%%%%%%%%%

{%
\usebackgroundtemplate{%
    \tikz\node[opacity=0.2]{%
        \includegraphics<handout:0>[height=\paperheight,width=\paperwidth]{%
            glass.jpg}};}%
\begin{frame}{Fragility}

    How it happens
    \begin{itemize}
        \item<1-> Implicit dependencies
        \item<2-> Relying on implementation details
        \item<3-> Relying upon side effects of operations
        \item<4-> Reaching past abstraction layers
        \item<5-> Unmanaged complexity
    \end{itemize}
\end{frame}
}

%%%%%%%%%%%%%%%%%%%%%%%%%%%%%%%%%%%%%%%%%%%%%%%%%%%%%%%%%%%%%%%%%%%%%%%%%%%%%%%%

{%
\usebackgroundtemplate{%
    \tikz\node[opacity=0.2]{%
        \includegraphics<handout:0>[height=\paperheight,width=\paperwidth]{%
            glass.jpg}};}%
\begin{frame}{Fragility}

    How to avoid it
    \begin{itemize}
        \item<1-> Implicit dependencies
        \item<2-> Law of Demeter: principle of least knowledge
        \item<3-> Avoid side effects, and don't rely on the side effects of
            other modules
        \item<4-> Rely on the published API
        \item<5-> Invent and \textbf{simplify}
    \end{itemize}
\end{frame}
}

%%%%%%%%%%%%%%%%%%%%%%%%%%%%%%%%%%%%%%%%%%%%%%%%%%%%%%%%%%%%%%%%%%%%%%%%%%%%%%%%

{%
\usebackgroundtemplate{%
    \tikz\node[opacity=1.0]{%
        \includegraphics<handout:0>[height=\paperheight,width=\paperwidth]{%
            boot.jpg}};}%
\begin{frame}<handout:0>{Immobility}
\end{frame}
}

{%
\usebackgroundtemplate{%
    \tikz\node[opacity=0.2]{%
        \includegraphics<handout:0>[height=\paperheight,width=\paperwidth]{%
            boot.jpg}};}%
\begin{frame}{Immobility}

    \begin{itemize}
        \item \href{https://www.merriam-webster.com/dictionary/immobile}
            {Incapable of being moved}
        \item<2-> Software for which extra effort is required in order to reuse.
    \end{itemize}
\end{frame}
}

%%%%%%%%%%%%%%%%%%%%%%%%%%%%%%%%%%%%%%%%%%%%%%%%%%%%%%%%%%%%%%%%%%%%%%%%%%%%%%%%

{%
\usebackgroundtemplate{%
    \tikz\node[opacity=0.2]{%
        \includegraphics<handout:0>[height=\paperheight,width=\paperwidth]{%
            boot.jpg}};}%
\begin{frame}{Immobility}
    How it happens
    \begin{itemize}
        \item<1-> Direct dependency on things you don't own
        \item<2-> Too many responsibilities
    \end{itemize}
\end{frame}
}

%%%%%%%%%%%%%%%%%%%%%%%%%%%%%%%%%%%%%%%%%%%%%%%%%%%%%%%%%%%%%%%%%%%%%%%%%%%%%%%%

{%
\usebackgroundtemplate{%
    \tikz\node[opacity=0.2]{%
        \includegraphics<handout:0>[height=\paperheight,width=\paperwidth]{%
            boot.jpg}};}%
\begin{frame}{Immobility}
    How it happens
    \begin{itemize}
        \item<1-> Depend upon the concept, not the details
        \item<2-> Reduce responsibilities to solve distinct problems
    \end{itemize}
\end{frame}
}

%%%%%%%%%%%%%%%%%%%%%%%%%%%%%%%%%%%%%%%%%%%%%%%%%%%%%%%%%%%%%%%%%%%%%%%%%%%%%%%%

{%
\usebackgroundtemplate{%
    \tikz\node[opacity=1.0]{%
        \includegraphics<handout:0>[height=\paperheight,width=\paperwidth]{%
            chaos.jpg}};}%
\begin{frame}<handout:0>{Viscosity}
\end{frame}
}

{%
\usebackgroundtemplate{%
    \tikz\node[opacity=0.2]{%
        \includegraphics<handout:0>[height=\paperheight,width=\paperwidth]{%
            chaos.jpg}};}%
\begin{frame}{Viscosity}

    \begin{itemize}
        \item \href{https://www.merriam-webster.com/dictionary/viscous}
            {Having or characterized by a high resistance to flow}
        \item<2-> Software for which extra effort is required in order to reuse.
    \end{itemize}
\end{frame}
}

%%%%%%%%%%%%%%%%%%%%%%%%%%%%%%%%%%%%%%%%%%%%%%%%%%%%%%%%%%%%%%%%%%%%%%%%%%%%%%%%

{%
\usebackgroundtemplate{%
    \tikz\node[opacity=0.2]{%
        \includegraphics<handout:0>[height=\paperheight,width=\paperwidth]{%
            chaos.jpg}};}%
\begin{frame}{Viscosity}
    Code that takes effort to maintain correctly

    \begin{itemize}
        \item<2->Viscous Design
        \begin{itemize}
            \item<4->When changing, preserving the design is difficult
        \end{itemize}
        \item<3->Viscous Environment
        \begin{itemize}
            \item<5->Long builds
            \item<6->Slow Tests
        \end{itemize}
    \end{itemize}
\end{frame}
}

%%%%%%%%%%%%%%%%%%%%%%%%%%%%%%%%%%%%%%%%%%%%%%%%%%%%%%%%%%%%%%%%%%%%%%%%%%%%%%%%

\section{Class Design}

%%%%%%%%%%%%%%%%%%%%%%%%%%%%%%%%%%%%%%%%%%%%%%%%%%%%%%%%%%%%%%%%%%%%%%%%%%%%%%%%

\begin{frame}{Principles of Object Oriented Class Design}
    SOLID Principles
    \begin{itemize}
        \item Single Responsibility Principle (SRP)
        \item Open Closed Principle (OCP)
        \item Liskov Substitution Principle (LSP)
        \item Interface Segregation Principle (ISP)
        \item Dependency Inversion Principle (DIP)
    \end{itemize}
\end{frame}


\section{Package Design}

%%%%%%%%%%%%%%%%%%%%%%%%%%%%%%%%%%%%%%%%%%%%%%%%%%%%%%%%%%%%%%%%%%%%%%%%%%%%%%%%

\begin{frame}{Principles of Package Architecture}
    \begin{itemize}
        \item<1-> Package Cohesion
            \begin{itemize}
                \item<3-> Release Reuse Equivalency Principle (REP)
                \item<4-> Common Closure Principle (CCP)
                \item<5-> Common Reuse Principle (CRP)
            \end{itemize}
        \item<2-> Package Coupling
            \begin{itemize}
                \item<6-> Acyclic Dependencies Principle (ADP)
                \item<7-> Stable Dependencies Principle (SDP)
                \item<8-> Stable Abstractions Principle (SAP)
            \end{itemize}
    \end{itemize}
\end{frame}

%%%%%%%%%%%%%%%%%%%%%%%%%%%%%%%%%%%%%%%%%%%%%%%%%%%%%%%%%%%%%%%%%%%%%%%%%%%%%%%%

\begin{frame}[t]\frametitle{title}



\end{frame}

\section{Architecture Design}

%%%%%%%%%%%%%%%%%%%%%%%%%%%%%%%%%%%%%%%%%%%%%%%%%%%%%%%%%%%%%%%%%%%%%%%%%%%%%%%%

\begin{frame}{Principles of Package Architecture}
\end{frame}

\section{Conclusion}
%%%%%%%%%%%%%%%%%%%%%%%%%%%%%%%%%%%%%%%%%%%%%%%%%%%%%%%%%%%%%%%%%%%%%%%%%%%%%%%%

\begin{frame}{Principles of Package Architecture}
\end{frame}

%%%%%%%%%%%%%%%%%%%%%%%%%%%%%%%%%%%%%%%%%%%%%%%%%%%%%%%%%%%%%%%%%%%%%%%%%%%%%%%%

\begin{frame}{References}
    \begin{itemize}
        \item \url{https://fi.ort.edu.uy/innovaportal/file/2032/1/design_principles.pdf}
        \item \url{http://www.butunclebob.com/ArticleS.UncleBob.PrinciplesOfOod}
        \item \url{http://notherdev.blogspot.com/2013/07/code-smells-rigidity.html}
        \item \url{https://dev.to/bob/how-do-you-know-your-code-is-bad}
        \item \url{http://staff.cs.utu.fi/~jounsmed/doos_06/slides/slides_060321.pdf}
        \item \url{https://softwareengineering.stackexchange.com/questions/357127/clear-examples-for-code-smells}
    \end{itemize}
\end{frame}

%%%%%%%%%%%%%%%%%%%%%%%%%%%%%%%%%%%%%%%%%%%%%%%%%%%%%%%%%%%%%%%%%%%%%%%%%%%%%%%%

\begin{frame}{Questions}

\end{frame}

\end{document}

